\documentclass{article}
\usepackage{ctex}
\usepackage{tabu}
\usepackage{fancyhdr, ulem}
\usepackage{geometry}
\usepackage{amsmath}
\usepackage{graphicx}
\usepackage{amsthm}
\usepackage{amsfonts}
\usepackage{keyval}
\usepackage[dvipsnames,svgnames,x11names]{xcolor}
\usepackage{float}
\usepackage{ifthen}
\usepackage{calc}
\usepackage{ifplatform}
\usepackage{fancyvrb}
\usepackage{hyperref}
\usepackage{amssymb}
\usepackage{multicol}
\usepackage{multirow}
\usepackage[all]{xy}
\usepackage{ulem}
\usepackage{verbatim}
\usepackage{indentfirst}
\usepackage{fancyhdr}
\usepackage{enumitem}
\setenumerate[1]{itemsep=0pt,partopsep=0pt,parsep=\parskip,topsep=5pt}
\setitemize[1]{itemsep=0pt,partopsep=0pt,parsep=\parskip,topsep=5pt}
\setdescription{itemsep=0pt,partopsep=0pt,parsep=\parskip,topsep=5pt}
\pagestyle{fancy}
\fancyhf{}
\fancyhead[L]{NOIP2018 Simulation}%上面左边
\fancyhead[R]{Hany01}%上面右边
\fancyfoot[C]{\thepage}%下面中间页码
\renewcommand{\headrulewidth}{0.8pt}%页眉下横线宽度
\renewcommand{\footrulewidth}{0pt}%
\begin{document}
    \title{NOIP2018 Simulation \\ Solution}
    \author{Hany01}
	\newpage
	\maketitle
	\sout{首先为我出题不好道歉。由于是第一次出题,可能有很多不尽人意的地方,希望大家谅解。}

		\section*{逛公园}
		我们先DFS出图的每一个环,得到环上编号最小和最大的节点,那么合法的区间一定不能同时跨过这两个点。于是我们搞出一个$R$数组,$R[i]$表示以$i$作为左端点时右端点最右可以到哪个位置。$R$数组显然是单调递增的。

		对于询问$l,r$,我们二分出一个位置$i$满足$R[l\dots i-1]\le r$,$R[i\dots r]\ge r$,直接计算即可。


		\section*{风筝}
		改变了一个值后,序列的LIS有两种情况:
		\begin{itemize}
			\item 序列的LIS不包含这个位置,那么答案就是原序列的LIS或者原序列的LIS-1(取决于该位置是否为LIS方案中一定包含的点)。
			\item 序列的LIS包含这个位置,分别从前往后、从后往前建两棵主席树即可。
		\end{itemize}

		如何判断原序列的LIS是否一定包含这个点?首先得到它在LIS中排第几个,然后统计有多少个排名与其相同,如果没有,那么一定包含。

		主席树不是NOIP考点?离线+树状数组就行了。

		\sout{其实这题数据范围本来是5e5,但我打gen时把范围打成4e5了又不想重造数据。。。就这样吧。}


		\section*{种花}
		$f_{i,j}$表示一共有$(i+j)$个元素,其中$j$个没有限制,另外$i$个被禁止放入一个位置。

那么有转移:
$$f_{0,0}=1$$
$$f_{x,0}=(x-1)(f_{x-1,0} + f_{x-2,0})$$
$$f_{x,y}=f_{x,y - 1} \times y + x \times f_{x - 1, y}$$

考虑新加入一个没有限制的元素。如果该元素放置的位置是有限制的,那么以前有限制的元素就变成了没有限制的了,有$x$种选择方法,剩下的元素的放置方案数为$f_{x-1,y}$;如果该元素放置的位置是没有限制的,那么有$y-1$种选择方法,剩下的元素的放置方案数为$f_{x,y-1}$。


分情况讨论:

1. 若$i$在$p_j$上,$j$在$p_i$上,那么有$f_{n-2,0}$种情况。贡献为$a_1=\max(0,p_j-p_i)$。

2. 若上面两个条件只满足一个,那么有$f_{n-3,1}$种情况。贡献为$a_2=\sum\limits_{k=1}^{p_j-1}k+\sum\limits_{k=1}^{n-p_i}k-2a_1$。(分别考虑$j$在$p_i$和$i$在$p_j$并减去$i,j$同时在$p_j,p_i$的情况)

3. 两个条件都不满足的有$f_{n-4,2}$种情况。贡献为$\sum\limits_{k=2}^n (n-k+1)k - a_1 - a_2 - \sum\limits_{k=1}^{p_i-1}k-\sum\limits_{k=1}^{n-p_j}k+\max(0,p_i-p_j)$。

所以答案为$\sum\limits_{j>i}(j-i)(a_1 f_{n-2,0} + a_2f_{n-3,1} + a_3 f_{n-4,2})$。


\end{document}
